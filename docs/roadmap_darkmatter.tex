\documentclass[11pt]{article}
\usepackage[utf8]{inputenc}
\usepackage{hyperref}
\usepackage{amsmath}
\usepackage{geometry}
\geometry{margin=1in}
\hypersetup{
  colorlinks=true,
  linkcolor=blue,
  urlcolor=cyan,
  pdftitle={Galactic Rotation Curves Simulator Roadmap},
  pdfauthor={Johnny Nguyen}
}

\title{\bfseries Galactic Rotation Curves Simulator\\\Large 20‑Week Roadmap with Dark‑Matter Extension}
\author{Johnny Nguyen \\ \small B.S. Computer Science ’27, Purdue University}
\date{\today}

\begin{document}
\maketitle
\begin{center}
  \textit{Driven by a passion for mathematics, physics, and discovery in astronomy}
\end{center}
\begin{center}
  \href{https://github.com/jnguyen727/galactic-rotation-curves}{GitHub Repository} \quad\textbullet\quad
\end{center}

\tableofcontents
\newpage

\section{Overview}
This document lays out a rigorous, 20‑week plan to build an interactive 3D web app for simulating galactic rotation curves, including a research‑grade Dark‑Matter module.  It covers fundamental derivations, algorithm implementations, custom shaders, UI controls, documentation, and outreach deliverables.

\section{Week 0: Project Setup \& Lab Notebook}
\textbf{Objectives:}
\begin{itemize}
  \item Initialize GitHub repo with CI (Vercel/GitHub Pages).
  \item Scaffold a lab‑notebook in Markdown with MathJax/KaTeX.
  \item Draft Project Charter: scope, goals, timeline, key references.
\end{itemize}
\textbf{Deliverables:}
\begin{itemize}
  \item Repo with \texttt{README.md}, \texttt{/notebook/week0.md}, and CI config.
\end{itemize}

\section{Weeks 1--2: Vector Math \& Rotations}
\textbf{Readings:} Arfken \& Weber §§1–3; Lengyel, \emph{3D Math Primer} Ch.~1–3.  
\textbf{Exercises:}
\begin{itemize}
  \item Derive dot/cross‑product identities and Rodrigues' formula.
  \item Implement \texttt{Vector3}, \texttt{rotateAxis(v,axis,\(\theta\))}, and \texttt{Quaternion} classes in JS.
\end{itemize}
\textbf{Deliverables:}
\begin{itemize}
  \item \texttt{vector3.js}, \texttt{rotation.js}, \texttt{quaternion.js} + unit tests.
  \item Lab notebook with LaTeX proofs (\texttt{notebook/week1.md}).
\end{itemize}

\section{Week 3: Kepler Orbits \& Conic Sections}
\textbf{Readings:} Goldstein Ch.~3; Kleppner \& Kolenkow App.~A.  
\textbf{Exercises:}
\begin{itemize}
  \item Derive polar conic equation \(r(\theta)=\frac{p}{1+e\cos\theta}\).
  \item Code \texttt{keplerOrbit(a,e,steps)} and plot in 2D.
\end{itemize}
\textbf{Deliverables:}
\begin{itemize}
  \item \texttt{keplerOrbit.js} + demo plot.
  \item Notebook derivation (\texttt{notebook/week2.md}).
\end{itemize}

\section{Weeks 4--5: Numerical Integrators}
\subsection{Week 4: RK4 Integrator}
\textbf{Readings:} Numerical Recipes Ch.~16; Hairer et al. Sec.~II.  
\textbf{Exercises:}
\begin{itemize}
  \item Derive RK4 update; implement \texttt{rk4Step()}.
  \item Simulate Sun–Earth; plot energy vs.\ time.
\end{itemize}
\subsection{Week 5: Adaptive RKF45 \& Leapfrog}
\textbf{Readings:} Hairer I Sec.~IV; \emph{Geometric Numerical Integration} Ch.~6.  
\textbf{Exercises:}
\begin{itemize}
  \item Implement RKF45 with error control.
  \item Derive and code velocity‑Verlet (symplectic leapfrog).
  \item Benchmark energy conservation vs.\ step size.
\end{itemize}
\textbf{Deliverables:}
\begin{itemize}
  \item \texttt{rk4.js}, \texttt{rkf45.js}, \texttt{leapfrog.js} + benchmark plots.
  \item Interactive error plot widget.
\end{itemize}

\section{Weeks 6--8: Stability, WebGL & Shaders}
\subsection{Week 6: Three‑Body \& Lagrange Points}
\textbf{Readings:} Binney \& Tremaine; Lagrange (1788).  
\textbf{Exercises:}
\begin{itemize}
  \item Derive \(L_{1\text{--}5}\) positions; implement \texttt{lagrange.js}.
  \item Build restricted 3‑body solver (Sun–Earth–Mars).
\end{itemize}
\subsection{Week 7: Raw WebGL Pipeline}
\textbf{Readings:} OpenGL Red Book Ch.~2–3; Rost GLSL Ch.~1–2.  
\textbf{Exercises:}
\begin{itemize}
  \item Write MVP vertex shader and fragment shader.
  \item Build helper functions (\texttt{initGL}, \texttt{compileShader}, \texttt{draw}) and render a cube.
\end{itemize}
\subsection{Week 8: Bloom & Custom Effects}
\textbf{Readings:} OpenGL ES 3.0; GPU Gems 3 Ch.~36.  
\textbf{Exercises:}
\begin{itemize}
  \item Implement off‑screen FBO + blur for bloom glow.
  \item Prototype grid/overlay shaders.
\end{itemize}

\section{Weeks 9--12: React‑Three‑Fiber & UI}
\subsection{Week 9: R3F Core \& Shaders}
\textbf{Exercises:}
\begin{itemize}
  \item Scaffold \texttt{<Canvas>} scene.
  \item Port raw shaders into \texttt{<shaderMaterial>}.
  \item Create \texttt{OrbitLine} component.
\end{itemize}
\subsection{Week 10: PBR \& Environment}
\textbf{Exercises:}
\begin{itemize}
  \item Use \texttt{meshPhysicalMaterial} with HDRI environment.
  \item Load textures for planets.
\end{itemize}
\subsection{Week 11: Post‑Processing}
\textbf{Exercises:}
\begin{itemize}
  \item Add \texttt{<EffectComposer>} with \texttt{<Bloom>} and fog.
\end{itemize}
\subsection{Week 12: Controls \& D3 Overlay}
\textbf{Exercises:}
\begin{itemize}
  \item Build React sliders for model parameters.
  \item Integrate D3 rotation‑curve plot overlay.
\end{itemize}

\section{Weeks 13--16: Dark‑Matter Rotation Curves}
\subsection{Week 13: Disk Curve}
\textbf{Derivation:}
\[
  v_{\rm disk}(r)
  = \sqrt{\frac{G}{r}\int_{0}^{r}2\pi\,\Sigma_{0} e^{-r'/R_d} r'\,dr'}.
\]
\textbf{Implementation:}
\begin{itemize}
  \item Write \texttt{rotationCurveDisk(rArray,Σ₀,R₍d₎)} + tests.
\end{itemize}
\subsection{Week 14: NFW Halo}
\textbf{Derivation:}
\[
  \rho(r)=\frac{\rho_{0}}{(r/R_{s})(1+r/R_{s})^{2}},
\quad
v_{\rm halo}(r)=\sqrt{\frac{G M_{\rm halo}(<r)}{r}}.
\]
\textbf{Implementation:}
\begin{itemize}
  \item Write \texttt{rotationCurveHalo(rArray,ρ₀,R₍s₎)}.
  \item Combine to get \(v_{\rm tot}(r)\).
\end{itemize}
\subsection{Week 15: Data Integration \& Plot}
\textbf{Exercises:}
\begin{itemize}
  \item Load `all_galaxies.json`; filter for selected galaxy.
  \item Plot \(v_{\rm obs}\), \(v_{\rm disk}\), and \(v_{\rm tot}\) with D3.
\end{itemize}
\subsection{Week 16: 3D Orbits}
\textbf{Exercises:}
\begin{itemize}
  \item Spawn particles at radii \(r_i\) with angular speed \(\omega_i = v_{\rm tot}(r_i)/r_i\).
  \item Sync 3D animation with 2D plot updates.
\end{itemize}

\section{Weeks 17--18: Documentation \& Outreach}
\begin{itemize}
  \item Embed KaTeX derivations in “Scholar Mode.”  
  \item Write a 12‑page technical report.  
  \item Publish a blog series and NPM packages.  
  \item Record a screencast and prepare slides.
\end{itemize}

\section{Week 19: Workshop \& Community}
\begin{itemize}
  \item Host a campus/online workshop.  
  \item Share on forums and collect feedback.
\end{itemize}

\section{Week 20: Reflection \& Extensions}
\begin{itemize}
  \item Analyze usage metrics.  
  \item Write a “Lessons Learned” post.  
  \item Plan future features (e.g.\ exoplanet ML, AR/VR).
\end{itemize}

\section{Data Sources \& References}
\begin{itemize}
  \item Lelli et al.\ (2016), “SPARC: Spitzer Photometry \& Accurate Rotation Curves.”  
  \item Navarro, Frenk \& White (1996), “The Structure of Cold Dark Matter Halos.”  
  \item Goldstein, H., \emph{Classical Mechanics}.  
  \item Press et al., \emph{Numerical Recipes}.  
  \item React‑Three‑Fiber, Drei, D3, KaTeX official docs.
\end{itemize}

\end{document}
